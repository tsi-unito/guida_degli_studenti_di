\chapter{Il PROLOG}

\dfn{PROLOG}{
  PROLOG (Programming Logic) è un \newfancyglitter{linguaggio dichiarativo} basato sul \newfancyglitter{paradigma logico}: 
  \begin{itemize}
    \item Non si descrive cosa fare per risolvere un problema. 
    \item Si descrive la situazione reale con \newfancyglitter{fatti} e \newfancyglitter{regole} e si chiede all'interprete di verificare se un \newfancyglitter{goal} segue oppure no secondo una logica classica.
  \end{itemize}
}

\nt{Il PROLOG è equivalente alla logica dei predicati del primordine.}

\section{Le Basi}

\dfn{Fatti}{Si rappresenta con dei \newfancyglitter{fatti} un dominio di interesse.}

\ex{Fatto}{
  Fatto per descrivere che un alimento contiene più calorie di un altro: 
  \begin{itemize}
    \item piuCalorico(wurstel, banana). 
    \item Rappresenta il fatto che il würstel è un alimento maggiormente
calorico rispetto alla banana.
  \end{itemize}
}

\dfn{Regole}{
  Si rappresentano le possibili inferenze con delle \newfancyglitter{regole}: 
  \begin{center}
    \texttt{head := subgoal1, subgoal2, \dots, subgoaln}
  \end{center}
}

\ex{Regola}{
  \begin{center}
    \texttt{felino(X) := gatto(X)}
  \end{center}
  Rappresenta la regola che permette di concludere che i gatti sono felini.
}

\paragraph{Idee di base del PROLOG:}

\begin{itemize}
  \item Regole ricorsive.
  \item L'interprete analizza i fatti e le regole nell'ordine in cui si trovano nel programma. 
  \item Meccanismo di pattern matching per uni care
variabili e termini. 
\item L’interprete, dato un programma, cerca di
dimostrare un goal considerando fatti e applicando
regole, nel secondo caso generando sotto-goal.
\end{itemize}

\dfn{Clausole}{
  Le clausole sono i fatti o le regole. Contengono:
  \begin{itemize}
    \item Atomi: 
      \begin{itemize}
        \item Costanti. 
        \item Numeri.
      \end{itemize}
    \item Variabili.
    \item Termini Composti, ottenuti applicando funtori a termini.

  \end{itemize}
}

\nt{Un programma PROLOG è un insieme di clausole.}

\clm{}{}{
  \begin{itemize}
    \item L'estensione dei file PROLOG è 'pl'.
    \item In PROLOG le variabili hanno l'iniziale maiuscola. 
    \item L'unica struttura dati nativa è la lista.
    \item Per eseguire swi: swipl. 
    \item Per compilare: $[$'nomefile.pl'$]$.
    \item Il comando ';' indica possibili alternative.
    \item Il comando 'trace.' consente un esecuzione passo per passo.
    \item '\textbackslash +' rappresenta la negazione per fallimento. 
    \item L'ordine è importante perché PROLOG "legge" dall'alto verso il basso.
\end{itemize}
}

\paragraph{Qualche predicato \fancyglitter{built-in}:}

\begin{itemize}
  \item \texttt{var(X)}: indica se \texttt{X} è una variabile. 
\item \texttt{ground(X)}: indica se \texttt{X} è istanziata. 
\item \texttt{atom(X)}: indica se \texttt{X} è atomica. 
\end{itemize}

\subsection{Liste}

\dfn{Lista}{
  La \newfancyglitter{lista} è la struttura dati principale in PROLOG. Una lista è caratterizzata da una testa e da una
coda: 
\begin{itemize}
  \item Testa: primo termine (a sinistra) della lista. 
  \item Coda: la lista dei termini dal secondo (incluso)
in poi.
\end{itemize}
}

\nt{Rappresentata come $[$Head $|$ Tail$]$.}

\begin{figure}[h]
    \centering
    \includegraphics[scale=0.6]{01/liste.png}
    \caption{Le liste in PROLOG.}
\end{figure}

\paragraph{Predicati \fancyglitter{built-in}:}

\begin{itemize}
  \item \texttt{length(Lista, N)}: ha successo se la \texttt{Lista} contiene \texttt{N} elementi. 
  \item \texttt{member(Elemento,Lista)}: ha successo se la \texttt{Lista} contiene il termine \texttt{Elemento}.
  \item \texttt{select(Elemento,Lista,Rimanenti)}: rimuove \texttt{Elemento} da \texttt{Lista} e restituisce \texttt{Rimanenti}. 
\end{itemize}

\section{Interprete PROLOG}

\qs{}{Come avviene l'esecuzione di programmi PROLOG?}

\begin{itemize}
  \item Esecuzione mediante \fancyglitter{backward chaining} in profondità. 
  \item Si parte dal \fancyglitter{goal} che si vuole derivare: 
    \begin{itemize}
      \item \fancyglitter{Goal} = congiunzione di formule atomiche $G_1, G_2, \dots, G_n$. 
      \item Si vuole dimostrare, mediante risoluzione, che il goal segua logicamente dal programma. 
    \end{itemize}
  \item Una regola $A :- B_1, B_2, \dots, B_m$ è applicabile a $G_i$ se: 
    \begin{itemize}
      \item Le variabili vengono rinominate. 
      \item $A$ e $G_i$ unificano. 
    \end{itemize}
\end{itemize}

\begin{figure}[h]
    \centering
    \includegraphics[scale=0.4]{02/Interprete PROLOG.png}
    \caption{Una formulazione non deterministica di come funziona l'interprete PROLOG.}
\end{figure}

\nt{MGU è il Most General Unifier: minimo sforzo per rendere uguali due variabili (il fatto e il goal).}

\begin{itemize}
  \item La computazione ha successo se esiste una computazione che
termina con successo. 
\item Non determinismo: non è specificata la regola scelta in R. 
\item Ma l'interprete PROLOG si comporta in modo \fancyglitter{deterministico}: 
  \begin{itemize}
    \item Le clausole vengono considerate nell’ordine in cui sono scritte
nel programma. 
\item Viene fatto backtracking all’ultimo punto di scelta ogni volta
che la computazione fallisce.
  \end{itemize}
\item In caso di successo, l’interprete restituisce una sostituzione per le
variabili che compaiono nel goal.
\end{itemize}

\subsection{Breve Ripasso di Logica}

\dfn{Logica Classica}{
Conseguenza logica definita semanticamente: dato una teoria e una formula, diciamo che la formula segue dalla
teoria se essa è vera in tutti i modelli della teoria.
}

\ex{Gatti}{
  \begin{itemize}
    \item I gatti miagolano: gatto $\rightarrow$ miagola. 
    \item I persiani sono gatti: persiano $\rightarrow$ gatto.
    \item Si vuole dimostrare che i persiani miagolano: k $\vDash$ persiano $\rightarrow$ miagola.
  \end{itemize} 
  \begin{center}
    \includegraphics[scale = 0.4]{02/gatto.png}
  \end{center}
}

\begin{itemize}
  \item Tuttavia il processo è molto laborioso già con poche formule e basi di conoscenza piccole. 
  \item Metodo di prova: procedura/algoritmo che calcola/dimostra se una formula è
conseguenza logica della teoria. 
\begin{itemize}
  \item \fancyglitter{Corretto}: se l’algoritmo dimostra F da K, allora F è
conseguenza logica di K. 
\item \fancyglitter{Completo}: se F è conseguenza logica di K, allora l’algoritmo
dimostra F da K.
\end{itemize}
\end{itemize}

\paragraph{Risoluzione:}

\begin{itemize}
  \item Si applica a formule in forma di \fancyglitter{clausole} (disgiunzioni di letterali\footnote{Formule atomiche o negazione di formule atomiche.}). 
  \item Si basa su un'unica regola di inferenza: 
    \begin{itemize}
      \item Date due clausole $C_1=A_1 \lor \dots \lor A_n$ e $C_2=B_1 \lor \dots \lor B_m$. 
      \item Se ci sono due letterali $A_i$ e $B_j$ tali che $A_i = \neg B_j$, allora posso derivare la clausola \fancyglitter{risolvente} $A_1 \lor \dots A_{i-1} \lor A_{i+1} \lor \dots \lor A_n \lor B_1 \lor \dots B_{j-1} \lor B_{j+1} \lor \dots \lor B_m$. 
      \item Il risolvente è conseguenza logica di $C_1 \cup C_2$
    \end{itemize}
  \item Data una teoria (insieme di formule) $K$ e una formula $F$, dimostro
che $F$ è conseguenza logica di $K$ per refutazione (dimostrare che $K \cup \neg F$ è inconsistente). 
\item Si parte dalle clausole $K \cup \neg F$, risolvendo a ogni passo due
clausole e aggiungendo il risolvente all’insieme di clausole. 
\item Si conclude quando si ottiene la clausola vuota.
\end{itemize}

\ex{Risoluzione gatti}{
  \begin{center}
    \includegraphics[scale=0.5]{02/ris.png}
  \end{center}
}

\paragraph{Inoltre:}

\begin{itemize}
  \item Se le due clausole $C_1=A_1 \lor \dots \lor A_n$ e $C_2=B_1 \lor \dots \lor B_m$ contengono variabili, i due letterali $A_i$ e $B_j$ devono essere tali che si possa fare l’\fancyglitter{unificazione} tra i due: 
    \begin{itemize}
      \item Unificazione: sostituzione $\alpha$ di variabili con termini o uguaglianza di variabili affinché $A_i=\neg B_j$.
      \item Clausola risolvente $[A_1 \lor \dots A_{i-1} \lor A_{i+1} \lor \dots \lor A_n \lor B_1 \lor \dots B_{j-1} \lor B_{j+1} \lor \dots \lor B_m] \alpha$. 
      \item Le sostituzioni di $\alpha$ sono applicate a $A_1 \lor \dots A_{i-1} \lor A_{i+1} \lor \dots \lor A_n \lor B_1 \lor \dots B_{j-1} \lor B_{j+1} \lor \dots \lor B_m$.
    \end{itemize}
\end{itemize}

\begin{figure}[h]
    \centering
    \includegraphics[scale=0.7]{02/uni.png}
    \caption{Unificazione di due termini.}
\end{figure}

\nt{Per ragioni d'efficienza, PROLOG non fa \fancyglitter{occur check}, ossia una variabile X unifica con f(X).}

\subsection{Risoluzione SLD}

Per arrivare a un linguaggio di programmazione PROLOG si vuole una strategia efficiente. 

\dfn{Risoluzione SLD}{
  Linear resolution with Selection function for Definite clauses: 
  \begin{itemize}
    \item K con clausole \newfancyglitter{definite}:
      \begin{itemize}
        \item Clausole di Horn: al più un letterale non negato. 
        \item Strategia linear input: a ogni passo di risoluzione, una \newfancyglitter{variante} di
una clausola è sempre scelta nella K di partenza (programma)
mentre l’altra è sempre il risolvente del passo precedente (goal, la
negazione di F al primo passo). 
\item Variante: clausola con variabili rinominate.
      \end{itemize}
  \end{itemize}
}

\nt{\underline{\{\textit{\textbf{NON}}} LSD.}

\qs{}{Ma perché ci si limita alle clausole di Horn?}

\paragraph{Risposta:} si rimuove la parte "intuitiva" che non può essere implementata nel PROLOG. Inoltre le clausole di Horn garantiscono la completezza.

\paragraph{Derivazione SLD per un goal $G_0$ da un insieme di clausole K è:}

\begin{itemize}
  \item Una sequenza di clausole goal $G_0, G_1, \dots, G_n$. 
  \item Una sequenza di varianti di clausole di $K C_1, C_2, \dots, C_n$. 
  \item Una sequenza di MGU $\alpha_1, \alpha_2, \dots, \alpha_n$, tali che $G_{i+1}$ è derivato da $G_i$ e da $C_{i+1}$
    attraverso la sostituzione $\alpha_{i+1}$,
\end{itemize}

\paragraph{Tre possibili tipi di derivazioni:}

\begin{itemize}
  \item Successo se $G_n$ è vuoto (\texttt{true}). 
  \item Fallimento finito, se non è possibile derivare da $G_n$ alcun risolvente e $G_n$ non è vuoto (\texttt{false}).
  \item Fallimento infinito, se è sempre possibile derivare nuovi risolventi (loop infinito).
\end{itemize}

\paragraph{Due forme di non determinismo:}

\begin{itemize}
  \item Regola di calcolo per selezionare a ogni passo l’atomo $B_i$
del goal da unificare con una clausola. 
\item Scelta di quale clausola utilizzare a ogni passo di
risoluzione. 
\end{itemize}

\dfn{Regola di calcolo}{
Funzione che ha come dominio l’insieme dei
goal e per ogni goal seleziona un suo atomo.
}

\nt{La regola di calcolo non influenza correttezza e completezza del metodo di prova.}

\qs{}{Come si costruisce l'albero SLD?}

\paragraph{Data una regola di calcolo, è possibile rappresentare tutte le
derivazioni con un albero SLD:}

\begin{itemize}
  \item Nodo: goal. 
  \item Radice: goal iniziale $G_0$. 
  \item Ogni nodo $\leftarrow A_1, \dots, A_m, \dots, A_k$, dove $A_m$ è l’atomo
selezionato dalla regola di calcolo, ha un figlio per ogni
clausola A $\leftarrow B_1, \dots, B_k$ tale che $A$ e $A_m$ sono unificabili con
MGU $\alpha$. Il nodo figlio è etichettato con il goal $\leftarrow [A_1, \dots, A_{m-1}, B_1, \dots, B_k, A_{m+1}, \dots, A_k] \alpha$. Il ramo dal padre al figlio è
etichettato con $\alpha$ e con la clausola selezionata.
\end{itemize}

\paragraph{Scelte per rendere la strategia deterministica:}

\begin{itemize}
  \item Regola di computazione: \fancyglitter{leftmost} (viene sempre scelto il sottogoal più a sinistra). 
  \item Clausole considerate nell’\fancyglitter{ordine in cui sono scritte nel programma}. 
  \item Strategia di ricerca: \fancyglitter{in profondità con backtracking}. 
    \begin{itemize}
      \item Non è completa perché se una computazione che porta al successo si trova a destra di un ramo infinito l'interprete non la trova, perché entra, senza mai uscirne, nel ramo infinito.
    \end{itemize}
\end{itemize}

\nt{Cercare di mettere a destra le computazioni che possano produrre eventuali casini.}

\subsection{Il Cut}

\dfn{Cut}{
  Il \newfancyglitter{cut} è un predicato extra-logico che consente di modificare l'esecuzione dell'interprete PROLOG. CUT (!): 
  \begin{itemize}
    \item Predicato sempre vero. 
    \item Se eseguito blocca il backtracking.
  \end{itemize}
}

\nt{Si rischia di perdere la completezza, ma si guadagna molto in efficienza.}

\paragraph{Modello run-time dell'interprete PROLOG:}

\begin{itemize}
  \item Due stack: 
    \begin{itemize}
      \item Stack di \fancyglitter{esecuzione}: contiene i record di
attivazione (environment) dei predicati in
esecuzione. 
\item Stack di \fancyglitter{backtracking}: contiene l'insieme dei punti di scelta (choice-point).
    \end{itemize}
  \item In realtà c'è un solo stack, con alternanza di environment e choice-point.
\end{itemize}

\paragraph{Il cut:}

\begin{itemize}
  \item Rende definitive le scelte fatte nel corso della valutazione dall'interprete PROLOG (eliminazione di choice-point dallo stack di backtracking). 
  \item Altera il controllo del programma. 
  \item Perdità di dichiaratività.
\end{itemize}

\begin{figure}[h]
    \centering
    \includegraphics[scale=0.7]{02/cut.png}
    \caption{Esempio di cut che provoca la perdita di completezza.}
\end{figure}

\section{Strategie di Ricerca in PROLOG}

\paragraph{Un problema di ricerca è definito da:} 
    \begin{itemize}
    \item \fancyglitter{Stato iniziale}. 
    \item \fancyglitter{Insieme delle azioni} (azione: fa passare da uno stato all'altro). 
    \item Specifica degli obiettivi (goal).
    \item Costo di ogni azione.
    \end{itemize}

\nt{Non tutti i problemi hanno una naturale soluzione con la ricerca nello spazio degli stati.}

\subsection{Ricerca nello Spazio degli Stati}

Lo spazio degli stati definito implicitamente dallo stato iniziale con un insieme delle azioni, ossia l'insieme di tutti gli stati raggiungibili a partire da quello iniziale.

\dfn{Cammino}{
  Sequenza di stati collegati da una sequenza di azioni.
}

\cor{Costo di un Cammino}{
  Somma dei costi delle azioni che lo
compongono.
}

\nt{Se non si hanno dei costi espliciti si assume che siano tutti uguali (e. g. tutti 1).}

\dfn{Soluzione a un Problema}{
Cammino dallo stato iniziale ad uno stato goal.
}

\cor{Soluzione Ottima}{
Soluzione che ha il costo minimo tra tutte le soluzioni.
}

\nt{Non è detto che esista una soluzione. In generale possono esistere 0, 1 o più soluzioni.}

\paragraph{Stati rappresentati come termini:}

\begin{itemize}
  \item Dipendono dal problema da rappresentare: 
    \begin{itemize}
      \item Mondo dei blocchi: on(a,b), clear(c), ecc. 
      \item Puzzle dell'8: lista ordinata [3, 1, v, 4, 7, 8, 5, 6, 2].
    \end{itemize}
\end{itemize}

\paragraph{Azioni specificate tramite:}

\begin{itemize}
  \item Precondizioni: in quali stati un'azione può essere eseguita.
  \item Effetti.
  \item applicabile(AZ, S): l'azione AZ è eseguibile nello stato S. 
  \item trasforma(AZ, S, NUOVO\_S): se l'azione AZ è applicabile a S, lo stato NUOVO\_S è il risultato dell'applicazione di AZ allo stato S.
\end{itemize}

\subsection{Cammini (Labirinto)}

\paragraph{Specifiche:}

\begin{itemize}
  \item Trovare un cammino in una griglia rettangolare, con
ostacoli in alcune celle. 
\item Predicati num\_righe e num\_colonne speci cano la
dimensione della griglia. 
\item pos(Riga,Colonna) per rappresentare la posizione
dell’agente. 
\item occupata(pos(Riga,Colonna)) per rappresentategli ostacoli.
\end{itemize}

\paragraph{Azioni:}

\begin{itemize}
  \item Nord. 
  \item Sud. 
  \item Ovest. 
  \item Est. 
\end{itemize}

\paragraph{Azione applicabile quando la sua esecuzione non porta l’agente:}

\begin{itemize}
  \item Fuori dalla griglia. 
  \item In una cella occupata da un ostacolo.
\end{itemize}

\begin{figure}[h]
    \centering
    \includegraphics[scale=0.6]{02/lab.png}
    \caption{Esempio di labirinto.}
\end{figure}

\begin{figure}[h]
    \centering
    \includegraphics[scale=0.7]{02/op.png}
    \caption{Operazioni possibili.}
\end{figure}

\paragraph{Altri predicati extra-logici (asserzioni):}

\begin{itemize}
  \item \texttt{assert(Fatto(X))}: aggiunge fatti alla base di conoscenza. 
  \item Può essere inserito in una regola.
  \item È un predicato \fancyglitter{dinamico}.
  \item \texttt{asserta(Fatto(X))}: inserisce in testa (prima nell'ordine). 
  \item \texttt{assertz(Fatto(X))}: inserisce in coda (dopo nell'ordine).
  \item \texttt{retract(Fatto(X))}: rimuove un fatto dalla base di conoscenza (ATTENZIONE: vale solo per i fatti inseriti dinamicamente da assert/asserta/assertz).
  \item \texttt{retractall(Fatto(\_))}: rimuove tutti i predicati relativi al fatto.
\end{itemize}

\subsection{Strategie di Ricerca}

\dfn{Strategie non Informate}{
  Strategie che non fanno assunzioni particolari sul dominio.
}

\paragraph{Strategie non informate:}

\begin{itemize}
  \item \fancyglitter{Ricerca in profondità}: 
    \begin{itemize}
      \item Espande sempre per primo il nodo più distante dalla radice
dell’albero di ricerca. 
\item i può realizzare facilmente in Prolog sfruttando il
nondeterminismo del linguaggio. 
    \end{itemize}
  \item \fancyglitter{Ricerca a profondità limitata}:
    \begin{itemize}
      \item Come per la ricerca in profondità, ma utilizzando un parametro
che vincola la profondità massima oltre la quale i nodi non
vengono espansi.
    \end{itemize}
  \item \fancyglitter{Iterative deepening}: 
    \begin{itemize}
      \item Ripete la ricerca a profondità limitata, incrementando a ogni
passo il limite. 
\item Ottima nel caso di azioni dal costo unitario.
    \end{itemize}
  \item \fancyglitter{Ricerca in ampiezza}: 
    \begin{itemize}
      \item Coda di nodi. 
      \item A ogni passo, la procedura espande il nodo in testa
        alla coda (\texttt{findall}) generando tutti i suoi successori,
che vengono aggiunti in fondo alla coda. 
\item Garantita l’individuazione della soluzione ottima.
    \end{itemize}
  \item \fancyglitter{Ricerca in ampiezza su grafi}:
    \begin{itemize}
      \item Come la ricerca in ampiezza, ma considerando la lista
chiusa dei nodi già espansi. 
\item Prima di espandere un nodo, si veri ca che non
sia chiuso. 
\item Il nodo chiuso non viene ulteriormente espanso.
    \end{itemize}
\end{itemize}

\dfn{Strategie Informate}{
  Utilizzano una funzione euristica h(n)\footnote{Costo stimato del cammino più conveniente dal nodo n a uno stato finale.}. Si associa un costo a ciascuna azione e viene definita una funzione g(x)\footnote{Costo del cammino trovato dal nodo iniziale a n.}
}

\paragraph{Strategie Informate:}

\begin{itemize}
  \item \fancyglitter{Ricerca in profondità IDA*}:
    \begin{itemize}
      \item Come iterative deepening, ma con soglia stimata a ogni
passo in base all’euristica. 
\item Al primo passo la soglia è h(si), dove si è lo stato iniziale. 
\item A ogni iterazione, la soglia è il minimo f(n) = g(n) +
h(n) per tutti i nodi n che superavano la soglia al passo
precedente (backtracking). 
\item si usa \texttt{assert} per salvare f(n) in caso di fallimento.
    \end{itemize}
  \item \fancyglitter{Ricerca in ampiezza con stima A*}: 
    \begin{itemize}
      \item Ricerca in ampiezza su gra che tiene conto della
funzione euristica. 
      \item A ogni passo si estrae per l’espansione dalla coda il
nodo con minimo valore di f(n) = g(n) + h(n). 
\item I nodi già espansi non vengono più espansi.
    \end{itemize} 
\end{itemize}





