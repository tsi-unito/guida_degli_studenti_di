\chapter{Laboratorio 2}

\section{Identificatori delle entità}

Gli \textcolor{blue}{identificatori} delle entità servono per identificarle univocamente. Possono essere:
\begin{itemize}
    \item \textcolor{blue}{interni:} sono costituiti dagli attributi delle entità;
    \item \textcolor{blue}{esterni:} sono costituiti da attributi delle entità più entità esterne, tramite associazioni.
\end{itemize}

Gli identificatori sono rappresentati come pallini pieni.

\begin{center}
    \begin{tikzpicture}[auto]

  \node[entity] (studente) at (0,0) {Studente};
    \tikzstyle{knode}=[circle,draw=black,thick,inner sep=2pt]
    \tikzstyle{knode_i}=[circle,draw=black, fill = black,thick,inner sep=2pt]
    \node (n1) at (0:1.5cm) [knode_i] {};
    \node (n2) at (100:1.5cm) [knode] {};

\path (n1) edge [above right] node {Matricola} (studente);
\path (n2) edge [above right] node {Anno di iscrizione} (studente);
    \end{tikzpicture}
\end{center}

Se sono necessari più attributi si rappresentano con una sbarra con pallino nero sopra gli attributi necessari.

Ogni entità deve avere almeno un identificatore e ogni attributo che fa parte di un identificatore deve avere  cardinalità (1, 1).

\section{Generalizzazione}

La \textcolor{blue}{generalizzazione} mette in relazione una o più entità $E_1$, $E_2$, ..., $E_n$ con una entità E che le comprende come casi particolari:

\begin{itemize}
    \item E è \textcolor{blue}{generalizzazione} di $E_1$, $E_2$, ..., $E_n$;
    \item $E_1$, $E_2$, ..., $E_n$ sono \textcolor{blue}{specializzazioni} di E;
\end{itemize}

Ogni occorrenza di $E_1$, $E_2$, ..., $E_n$ è anche occorrenza di E. Ogni proprietà di E è anche proprietà di $E_1$, $E_2$, ..., $E_n$.

Una generalizzazione può essere:
\begin{itemize}
    \item \textcolor{blue}{totale} se ogni occorrenza dell'entità genitore è occorrenza di almeno una delle entità figlie;
    \item \textcolor{blue}{parziale} se non è totale;
    \item \textcolor{blue}{esclusiva} se ogni occorrenza dell'entità genitore è occorrenza di al più una delle entità figlie;
    \item \textcolor{blue}{sovrapposta} se non è esclusiva.
\end{itemize}

\section{Documentazione schemi concettuali}

\begin{itemize}
    \item \textcolor{blue}{Descrizione di concetti:} dizionari per entità e associazioni;
    \item \textcolor{blue}{Vincoli non esprimibili in ER:} di integrità o di derivazione.
\end{itemize}