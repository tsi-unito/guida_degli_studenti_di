\chapter*{\centering \texttt{Premessa}}

\section*{Licenza}

Questi appunti sono rilasciati sotto licenza Creative Commons Attribuzione 4.0 Internazionale (per maggiori
informazioni consultare il link: \href{https://creativecommons.org/version4/}{https://creativecommons.org/version4/}).
Sono basati sulle lezioni del corso "Architetture degli Elaboratori II" (2024-2025) dell'università di Torino: prof. Gunetti (i primi 5 capitoli) e del prof. Aldinucci (il sesto capitolo). Sono contenuti riferimenti a: 
\begin{itemize}
  \item [1.] J. L. Hennessy \& D. A. Patterson: Computer Architecture:
A Quantitative Approach, 6th edition. Elsevier - Morgan Kaufmann, 2019. 
\item [2.] A. S. Tanenbaum: Structured Computer Organization, 5th ed.
Pearson, 2005. 
\item [3.] NVIDIA CUDA SAMPLES.
\end{itemize}

\begin{center}
    \includegraphics[width=0.5\textwidth]{images/cc.png}
\end{center}

\section*{Formato utilizzato}

\subsubsection{Box di "Corollario":}

\cor{Nome del corollario}{
    Testo del corollario. Per corollario si intende una definizione minore,
    legata a un'altra definizione.
}

\subsubsection{Box di "Definizione":}

\dfn{Nome delle definizione}{
    Testo della definizione.
}

\subsubsection{Box di "Domanda":}

\qs{}{
    Testo della domanda. Le domande sono spesso utilizzate per far riflettere
    sulle definizioni o sui concetti.
}

\subsubsection{Box di "Esempio":}

\ex{Nome dell'esempio}{
    Testo dell'esempio. Gli esempi sono tratti dalle slides del corso.
}

\subsubsection{Box di "Note":}

\nt{
    Testo della nota. Le note sono spesso utilizzate per chiarire concetti
    o per dare informazioni aggiuntive.
}

\subsubsection{Box di "Osservazioni":}

\clm{}{}{Testo delle osservazioni. Le osservazioni sono spesso utilizzate
per chiarire concetti o per dare informazioni aggiuntive.
A differenza delle note le osservazioni sono più specifiche.}

