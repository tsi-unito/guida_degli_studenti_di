% tipo di documento
\documentclass[a4paper, twoside, openright]{report}
% codifica caratteri
\usepackage[utf8]{inputenc}
% encoding del testo
\usepackage[T1]{fontenc}
% dimensione dei margini
\usepackage[a4paper,top=2.5cm,bottom=2.5cm,left=3cm,right=3cm]{geometry}
% dimensione del font
\usepackage[fontsize=12pt]{scrextend}
% lingua del testo
\usepackage[english,italian]{babel}
% lingua per la bibliografia
\usepackage[fixlanguage]{babelbib}
% package per generare testo fittizio. Potrebbe essere
% utile nel controllare quanto un capitolo potrebbe essere
% grande e quindi quanto occupa nella pagina
\usepackage{lipsum}
% per ruotare le immagini
\usepackage{rotating}
% per modificare l'header delle pagine 
\usepackage{fancyhdr}
% per allineare in modo giustificato
\usepackage{ragged2e}
\justifying
% uso delle immagini
\usepackage{graphicx}
\usepackage{float}
% uso dei colori
\usepackage[dvipsnames, table]{xcolor}         
% uso dei listing per il codice
\usepackage{listings}          
% per inserire gli hyperlinks tra i vari elementi del testo 
\usepackage[colorlinks=true, allcolors=black]{hyperref}    
% diversi tipi di sottolineature
\usepackage[normalem]{ulem}
% package e comando per creare pagine vuote
\usepackage{afterpage}
\newcommand\blankpage{%
    \null
    \thispagestyle{empty}%
    \addtocounter{page}{-1}%
    \newpage
}
    
% package per creare comandi personalizzati
\usepackage{xpatch}
% package helper per le liste puntate
\usepackage{enumitem}
% package per l'utilizzo dei colori
% package per l'highlighting del codice
\usepackage{minted}
% package per gestire le caption
\usepackage{caption}
\usepackage{subcaption}
% per gestire tabelle su più pagine
\usepackage{longtable}
% per combinare le righe di una tabella
\usepackage{multirow}
% per creare i tree di directory
\usepackage{dirtree}

% per le icone a fianco dei titoli di sezione
\usepackage{etoolbox}
\newcommand{\icon}[1]{\includegraphics[height=12pt]{#1}}
\robustify{\icon}

% -----------------------------------------------------------------

% Modifica lo stile dell'header
\pagestyle{fancy}
\fancyhf{}
\lhead{\rightmark}
\rhead{\textbf{\thepage}}
\fancyfoot{}
\setlength{\headheight}{15pt}

% Rimuove il numero di pagina all'inizio dei capitoli
\fancypagestyle{plain}{
  \fancyfoot{}
  \fancyhead{}
  \renewcommand{\headrulewidth}{0pt}
}

% comandi per cambiare temporaneamente la lingua
% abstract in inglese, al fine di cambiarne il titolo
\xpretocmd{\abstract}{\selectlanguage{english}}{}{} 
\xapptocmd{\endabstract}{\selectlanguage{italian}}{}{}

% formattazione e highlight del codice
\usemintedstyle{manni}
\setminted[typescript]{
    framesep=2mm,
    baselinestretch=1.2,
    fontsize=\ttfamily\footnotesize,
    linenos
}

% rimozione del prefix per le tabelle
\captionsetup[table]{labelformat=empty}

% environment per impostare il codice in piu' pagine
\newenvironment{code}{\captionsetup{type=listing}}{}

% -----------------------------------------------------------------
\begin{document}

\begin{titlepage}

\begin{center}
    \textbf{\huge{Università degli Studi di Torino}}
    \vspace{2mm}
    \\ \LARGE{Dipartimento di informatica}
    \vspace{5mm}
    \\ \includegraphics[keepaspectratio=true,scale=0.8]{images/unito_logo.pdf}
    \vspace{5mm}
\end{center}

\begin{center}
    \LARGE{Tesi di Laurea Triennale in Informatica} 
\end{center}

\vspace{15mm}
\begin{center}
    \textbf{\huge{ A beautiful title for your Bachelor's Thesis }}
\end{center}
\vspace{30mm}

\begin{minipage}[t]{0.47\textwidth}
	{\large{Relatore:}{\normalsize\vspace{3mm}
	\bf\\ \large{Prof: Name Surname} \normalsize\vspace{3mm}\bf}}
\end{minipage}
\hfill
\begin{minipage}[t]{0.47\textwidth}\raggedleft
	{\large{Candidato:}{\normalsize\vspace{3mm} \bf\\ \large{Name Surname}}}
\end{minipage}

\vspace{40 mm}
\hrulefill
\\ \centering{\large{ANNO ACCADEMICO XXXX/XXXX}}

\end{titlepage}
\afterpage{\blankpage}
\thispagestyle{plain}

{
\raggedleft
\textit{A very cool quote!}
\\ - Author of the quote

}
\afterpage{\blankpage}
\begin{abstract}
    \lipsum[1-4]
\end{abstract}
\afterpage{\blankpage}
\thispagestyle{plain}
\vspace*{\fill}
\textit{Dichiaro di essere responsabile del contenuto dell’elaborato che presento al fine del conseguimento del titolo, di non avere plagiato in tutto o in parte il lavoro prodotto da altri e di aver citato le fonti originali in modo congruente alle normative vigenti in materia di plagio e di diritto d’autore. Sono inoltre consapevole che nel caso la mia dichiarazione risultasse mendace, potrei incorrere nelle sanzioni previste dalla legge e la mia ammissione alla prova finale potrebbe essere negata.}
\vspace*{\fill}

\tableofcontents

\chapter{Chapter Title}

An introduction here (with a cite from bibliography, like this: \cite{greenwade93}).
%  ----
\section{Type of texts}
\subsection{Normal text}
\lipsum[1]

\subsection{Normal text}
\textbf{\lipsum[1]}

\subsection{Italic text}
\textit{\lipsum[1]}

\subsection{Mono space text}
\subsubsection{(usually I use this type of format to refer me to code)}
\texttt{Here is some text. Pay attention to this type, because It can go 
\\ out of bounds! Usually you solve it with a double back-slash, to 
\\ go to the new line.}
%  ----

%  ----
\section{Lists}
\subsection{Unbulleted lists}
\begin{itemize}
    \item \lipsum[1]
    \item \lipsum[1]
    \item \lipsum[1]
\end{itemize}

\subsection{Bulleted lists}
\begin{enumerate}
    \item \lipsum[1]
    \item \lipsum[1]
    \item \lipsum[1]
\end{enumerate}
%  ----

%  ----
\section{Images}
Random Duolingo Image example:
\begin{figure}[H]
\centering
    \includegraphics[scale=0.1]{images/duolingo.png}
    \caption{Caption example.}
    \label{fig:duolingo}
\end{figure}

In the text, you reference an image in this way: \ref{fig:duolingo}.
%  ----

%  ----
\section{Tables}
Example of table that can even be split into several pages if it's too long:
\begin{longtable}[H]{| l | l |}
    \hline
     \rowcolor[HTML]{F87C58}\textbf{Head 1} & \textbf{Head 2}\\
    \hline
    \endfirsthead
    
    \texttt{Text} & \texttt{Text}\\
    \hline
    \texttt{text} & \texttt{Text}\\
    \hline
    \texttt{Text} & \texttt{Text}\\
    \hline
    \texttt{text} & \texttt{Text}\\
    \hline
    \texttt{Text} & \texttt{Text}\\
    \hline
    \texttt{text} & \texttt{Text}\\
    \hline
    \texttt{Text} & \texttt{Text}\\
    \hline
    \texttt{text} & \texttt{Text}\\
    \hline
    \texttt{Text} & \texttt{Text}\\
    \hline
    \texttt{text} & \texttt{Text}\\
    \hline
    \texttt{Text} & \texttt{Text}\\
    \hline
    \texttt{text} & \texttt{Text}\\
    \hline
    \texttt{Text} & \texttt{Text}\\
    \hline
    \texttt{text} & \texttt{Text}\\
    \hline
    \texttt{Text} & \texttt{Text}\\
    \hline
    \texttt{text} & \texttt{Text}\\
    \hline
    \texttt{Text} & \texttt{Text}\\
    \hline
    \texttt{text} & \texttt{Text}\\
    \hline
    \texttt{Text} & \texttt{Text}\\
    \hline
    \texttt{text} & \texttt{Text}\\
    \hline
    \caption{Table caption.}
\end{longtable}
%  ----

%  ----
\section{Code}
Finally, here we are with code! The following examples are with typescript and HTML with Angular syntax, colored using the \textbf{minted} package.

\begin{code}
    \inputminted{typescript}{listings/ts-example.ts}
    \caption{Caption for the typescript example.}
\end{code}
\vspace{4mm}

\begin{code}
    \inputminted{ng2}{listings/html-example.html}
    \caption{Caption for the HTML with Angular syntax example.}
\end{code}
\vspace{4mm}
%  ----

\bibliographystyle{plain}
\bibliography{Bibliography}
\afterpage{\blankpage}
\thispagestyle{plain}
\textbf{There will be the thanks to precious people here.}

\lipsum[1-3]

\end{document}
% -----------------------------------------------------------------
